\documentclass{article}
\usepackage[utf8]{inputenc}
\usepackage{amsmath} % Esencial para matrices y símbolos matemáticos
\usepackage{amsfonts} % Para el símbolo de números Reales
\usepackage{geometry} % Para ajustar los márgenes
\usepackage{graphicx} % Para incluir imágenes si es necesario

\geometry{a4paper, margin=2cm} % Reduce los márgenes para más espacio

\title{Ejercicios Guía 1 - ALC}
\author{Vic}
\date{} % Fecha vacía

\begin{document}

\maketitle

\section*{Ejercicio 1}

\textit{Voy a resolver primero los sistemas de ecuaciones lineales no homogéneos, utilizando el \textbf{método de eliminación de Gauss.}}

\subsection*{a)}
\subsection*{Caso no homogéneo}

\[
\left(
\begin{array}{cccc|c}
1 & 1 & -2 & 1 & -2 \\
3 & -2 & 1 & 5 & 3\\
1 & -1 & 1 & 2 & 2
\end{array}
\right)
\xrightarrow[\substack{F_3 - F_1}]{F_2 - 3F_1}
\]

\[
\left(
\begin{array}{cccc|c}
1 & 1 & -2 & 1 & -2 \\
0 & -5 & 7 & 2 & 9\\
0 & -2 & 3 & 1 & 4
\end{array}
\right)
\xrightarrow[\substack{5F_3 - 2F_2}]{}
\]

\[
\left(
\begin{array}{cccc|c}
1 & 1 & -2 & 1 & -2 \\
0 & -5 & 7 & 2 & 9\\
0 & 0 & 1 & 1 & 2
\end{array}
\right)
\]

Ahora, despejamos las variables comenzando desde la última fila hacia la primera:
\begin{align*}
x_3 + x_4 &= 2 \quad \Rightarrow \quad x_3 = 2 - x_4 \\
-5x_2 + 7(2 - x_4) + 2x_4 &= 9 \quad \Rightarrow \quad -5x_2 + 14 - 7x_4 + 2x_4 = 9 \\
-5x_2 - 5x_4 &= -5 \quad \Rightarrow    \quad x_2 + x_4 = 1 \quad \Rightarrow \quad x_2 = 1 - x_4 \\
x_1 + (1 - x_4) - 2(2 - x_4) + x_4 &= -2 \\
x_1 + 1 - x_4 - 4 + 2x_4 + x_4 &= -2 \\
x_1 - 3 + 2x_4 &= -2 \quad \Rightarrow \quad x_1 = -2x_4 + 1
\end{align*}


Por lo tanto, la solución general del sistema es:\[
\begin{pmatrix}
-2x_4 + 1 \\
1 - x_4 \\
2 - x_4 \\
x_4 \\  
\end{pmatrix}   =
\begin{pmatrix}      
1 \\
1 \\ 
2 \\
0 \\        
\end{pmatrix} + x_4
\begin{pmatrix} 
-2 \\
-1 \\   
-1 \\
1 \\    
\end{pmatrix}, \quad x_4 \in \mathbb{R}
\]  

\subsection*{Caso homogéneo}

\[
\left(
\begin{array}{cccc|c}
1 & 1 & -2 & 1 & 0 \\
3 & -2 & 1 & 5 & 0\\
1 & -1 & 1 & 2 & 0
\end{array}
\right)
\xrightarrow[\substack{F_3 - F_1}]{F_2 - 3F_1}
\]

\[
\left(
\begin{array}{cccc|c}
1 & 1 & -2 & 1 & 0 \\
0 & -5 & 7 & 2 & 0\\
0 & -2 & 3 & 1 & 0
\end{array}
\right)
\xrightarrow[\substack{5F_3 - 2F_2}]{}
\]

\[
\left(
\begin{array}{cccc|c}
1 & 1 & -2 & 1 & 0 \\
0 & -5 & 7 & 2 & 0\\
0 & 0 & 1 & 1 & 0
\end{array}
\right)
\]


Ahora, despejamos las variables comenzando desde la última fila hacia la primera:
\begin{align*}
x_3 + x_4 &= 0 \quad \Rightarrow \quad x_3 = -x_4 \\
-5x_2 + 7(-x_4) + 2x_4      &= 0 \quad \Rightarrow \quad -5x_2 - 5x_4 = 0 \quad \Rightarrow    \quad x_2 + x_4 = 0 \quad \Rightarrow \quad x_2 = -x_4 \\
x_1 + (-x_4) - 2(-x_4) + x_4 &= 0 \\
x_1 - x_4 + 2x_4 + x_4 &=   0 \\                
x_1 + 2x_4 &= 0 \quad \Rightarrow \quad x_1 = -2x_4
\end{align*}    

Por lo tanto, la solución general del sistema homogéneo es:\[
\begin{pmatrix} 
-2x_4 \\
-x_4 \\ 
-x_4 \\
x_4 \\
\end{pmatrix}   =
x_4
\begin{pmatrix}
-2 \\
-1 \\   
-1 \\
1 \\
\end{pmatrix}, \quad x_4 \in \mathbb{R}
\]


\subsection*{b)}
\subsection*{Caso no homogéneo}
\[
\left(
\begin{array}{ccccc|c}
1 & 1 & 1 & -2 & 1 & 1 \\
1 & -3 & 1 & 1 & 1 & 0\\
3 & -5 & 3 & 0 & 3 & 0
\end{array}
\right)
\xrightarrow[\substack{F_3 - 3F_1}]{F_2 - F_1}
\]

\[
\left(
\begin{array}{ccccc|c}
1 & 1 & 1 & -2 & 1 & 1 \\
0 & -4 & 0 & 3 & 0 & -1\\
0 & -8 & 0 & 6 & 0 & -3
\end{array}
\right)
\xrightarrow[\substack{F_3 - 2F_2}]{}
\]

\[
\left(
\begin{array}{ccccc|c}
1 & 1 & 1 & -2 & 1 & 1 \\
0 & -4 & 0 & 3 & 0 & -1\\
0 & 0 & 0 & 0 & 0 & -1
\end{array}
\right)
\]

La última fila indica que el sistema es incompatible, ya que $0 = -1$ es una contradicción. Por lo tanto, el sistema no tiene solución.
Es un \textbf{sistema incompatible}.

\subsection*{Caso homogéneo}
\[
\left(
\begin{array}{ccccc|c}
1 & 1 & 1 & -2 & 1 & 0 \\
1 & -3 & 1 & 1 & 1 & 0\\
3 & -5 & 3 & 0 & 3 & 0
\end{array}
\right)
\xrightarrow[\substack{F_3 - 3F_1}]{F_2 - F_1}
\]

\[
\left(
\begin{array}{ccccc|c}
1 & 1 & 1 & -2 & 1 & 0 \\
0 & -4 & 0 & 3 & 0 & 0\\
0 & -8 & 0 & 6 & 0 & 0
\end{array}
\right)
\xrightarrow[\substack{F_3 - 2F_2}]{}
\]

\[
\left(
\begin{array}{ccccc|c}
1 & 1 & 1 & -2 & 1 & 0 \\
0 & -4 & 0 & 3 & 0 & 0\\
0 & 0 & 0 & 0 & 0 & 0
\end{array}
\right)
\]

Ahora , despejamos las variables comenzando desde la segunda fila hacia la primera:
\begin{align*}  
-4x_2 + 3x_4 &= 0 \quad \Rightarrow \quad -4x_2 = -3x_4 \quad \Rightarrow \quad x_2 = \frac{3}{4}x_4 \\
x_1 + \left(\frac{3}{4}x_4\right) + x_3 - 2x_4 + x_5 &= 0 \\
x_1 + x_3 + x_5 + \frac{3}{4}   
x_4 - 2x_4 &= 0 \quad \Rightarrow \quad x_1 + x_3 + x_5 - \frac{5}{4}x_4 = 0 \\
x_1 = -x_3 - x_5 + \frac{5}{4}x_4
\end{align*}    
Por lo tanto, la solución general del sistema homogéneo es:\[
\begin{pmatrix} 
-x_3 - x_5 + \frac{5}{4}x_4 \\
\frac{3}{4}x_4 \\ 
x_3 \\  
x_4 \\  
x_5 \\      
\end{pmatrix}   =
x_3
\begin{pmatrix}
-1 \\       
0 \\
1 \\
0 \\
0 \\
\end{pmatrix} +
x_4
\begin{pmatrix}
\frac{5}{4} \\
\frac{3}{4} \\
0 \\
1 \\
0 \\
\end{pmatrix} +
x_5
\begin{pmatrix}
-1 \\
0 \\
0 \\
0 \\
1 \\
\end{pmatrix}, \quad x_3, x_4, x_5 \in \mathbb{R}
\]


\subsection*{c)}
\subsection*{Caso no homogéneo}
\[
\left(
\begin{array}{ccc|c}
i & -(1+i) & 0 & -1\\
1 & -2 & 1 & 0\\
1 & 2i & -1 & 2i
\end{array}
\right)
\xrightarrow[\substack{iF_3 - F_1}]{iF_2 - F_1}
\] 

\[
\left(
\begin{array}{ccc|c}
i & -(1+i) & 0 & -1\\
0 & 1 - i & i & -1\\
0 & i - 1 & -i & -1
\end{array}
\right)
\xrightarrow[\substack{F_3 - F_2}]{}
\] 

\[
\left(
\begin{array}{ccc|c}
i & -(1+i) & 0 & -1\\
0 & 1 - i & i & -1\\
0 & 0 & -2i & 0
\end{array}
\right)
\] 

Ahora, despejamos las variables comenzando desde la última fila hacia la primera:
\begin{align*}
-2ix_3 &= 0 \quad \Rightarrow \quad x_3 = 0 \\
(1 - i)x_2 + ix_3 &= -1 \quad \Rightarrow \quad (1 - i)x_2 = -1 \quad \Rightarrow \\
\quad x_2 = \frac{-1}{1 - i} = \frac{-1(1 + i)}{(1 - i)(1 + i)} = \frac{-1 - i}{2} = -\frac{1}{2} - \frac{i}{2} \\        
ix_1 - (1 + i)x_2 &= -1 \quad \Rightarrow 
\quad ix_1 - (1 + i)\left(-\frac{1}{2} - \frac{i}{2}\right) = -1 \\
ix_1 + \frac{1}{2} + \frac{i}{2} + \frac{i}{2} + \frac{1}{2} &= -1 \\
ix_1 + 1 + i &= -1 \quad \Rightarrow \quad ix_1 = -1 - i - 1 = -2 - i \quad \Rightarrow \quad x_1 = \frac{-2 - i}{i} = \\
\frac{(-2 - i)(-i)}{i(-i)} = \frac{2i + i^2}{1} = \frac{2i - 1}{1} = -1 + 2i
\end{align*}


Por lo tanto, la solución general del sistema es:\[
\begin{pmatrix}                         
-1 + 2i \\
-\frac{1}{2} - \frac{i}{2} \\   
0 \\  
\end{pmatrix}   
\]          

\subsection*{Caso homogéneo}
\[
\left(
\begin{array}{ccc|c}
i & -(1+i) & 0 & 0\\
1 & -2 & 1 & 0\\
1 & 2i & -1 & 0
\end{array}
\right)
\xrightarrow[\substack{iF_3 - F_1}]{iF_2 - F_1}
\] 

\[
\left(
\begin{array}{ccc|c}
i & -(1+i) & 0 & 0\\
0 & 1 - i & i & 0\\
0 & i - 1 & -i & 0
\end{array}
\right)
\xrightarrow[\substack{F_3 - F_2}]{}
\] 

\[
\left(
\begin{array}{ccc|c}
i & -(1+i) & 0 & 0\\
0 & 1 - i & i & 0\\
0 & 0 & -2i & 0
\end{array}
\right)
\] 

Ahora, despejamos las variables comenzando desde la última fila hacia la primera:
\begin{align*}  
-2ix_3 &= 0 \quad \Rightarrow \quad x_3 = 0 \\
(1 - i)x_2 + ix_3 &= 0 \quad \Rightarrow \quad (1 - i)x_2 = 0 \quad \Rightarrow \quad x_2 = 0 \\        
ix_1 - (1 + i)x_2 &= 0 \quad \Rightarrow \quad ix_1 - (1 + i)(0) = 0 \quad \Rightarrow \quad ix_1 = 0 \quad \Rightarrow \quad x_1 = 0
\end{align*}

Por lo tanto, la única solución del sistema homogéneo es la trivial:\[
\begin{pmatrix}
0 \\
0 \\
0 \\
\end{pmatrix}
\]

\newpage

\section*{Ejercicio 2}

\includegraphics[scale=0.6]{Consignas-imagenes/image.png}

\subsection*{a)}
Primero empiezo triangularizando la matriz:
\[
\left(
\begin{array}{ccc|c}
1 & k & -1 & 1\\
-1 & 1 & k^2 & -1\\
1 & k & k-2 & 2
\end{array}
\right)
\xrightarrow[\substack{F_3 - F_1}]{F_2 + F_1}
\]

\[
\left(
\begin{array}{ccc|c}
1 & k & -1 & 1\\
0 & 1 + k & k^2 - 1 & 0\\
0 & 0 & k - 1 & 1
\end{array}
\right)
\]

Ahora despejo la última fila:
\begin{align*}
(k - 1) x_3 &= 1 \quad \Rightarrow \quad x_3 = \frac{1}{k - 1} \quad \text{si } k \neq 1\\
\end{align*}
pues si k = 1, la ecuación no tiene solución, me quedaría que 1=0 y eso es ABS.

Ahora sigo con la segunda fila:
\begin{align*}
(1 + k)x_2 + (k^2 - 1)x_3 &= 0 \\
\end{align*}
Que valor de k hace que 1 + k=0? k = -1.
Entonces, si k = -1:
\begin{align*}
0 + ((-1)^2 - 1)x_3 &= 0 \quad \Rightarrow \quad 0 = 0
\end{align*}
Por lo tanto, para k = -1 el sistema tiene infinitas soluciones con $x_2$ como parámetro libre.

Por otro lado, para que el sistema tenga solución única, necesito que k no sea ni 1 ni -1.

\subsection*{b)}

Triangulo la matriz homogénea:

\[
\left(
\begin{array}{ccc|c}
1 & k & -1 & 0\\
-1 & 1 & k^2 & 0\\
1 & k & k-2 & 0
\end{array}
\right)
\xrightarrow[\substack{F_3 - F_1}]{F_2 + F_1}
\]

\[
\left(
\begin{array}{ccc|c}
1 & k & -1 & 0\\
0 & 1 + k & k^2 - 1 & 0\\
0 & 0 & k - 1 & 0
\end{array}
\right)
\]

Si o si, para resolver el sistema y que no sea el resultado trivial, necesito que algun pivote sea cero, pruebo con:
 $k - 1 = 0 \quad \Rightarrow \quad k = 1$.

 $k + 1 = 0 \quad \Rightarrow \quad k = -1$.


Resuelvo para $k = 1$:
\[
\left(  
\begin{array}{ccc|c}
1 & 1 & -1 & 0\\
-1 & 1 & 1 & 0\\
1 & 1 & -1 & 0
\end{array}
\right)
\xrightarrow[\substack{F_3 - F_1}]{F_2 + F_1}
\]

\[
\left(  
\begin{array}{ccc|c}
1 & 1 & -1 & 0\\
0 & 2 & 0 & 0\\
0 & 0 & 0 & 0\\
\end{array}
\right)
\]

Despejo las variables:
\begin{align*}
2x_2 &= 0 \quad \Rightarrow \quad x_2 = 0 \\
x_1 + (0) - x_3 &= 0 \quad \Rightarrow
\quad x_1 = x_3
\end{align*}
Por lo tanto, la solución general del sistema homogéneo para $k = 1$ es:\[
\begin{pmatrix}
x_3 \\
0 \\
x_3 \\
\end{pmatrix}   =
x_3
\begin{pmatrix}
1 \\
0 \\
1 \\
\end{pmatrix}, \quad x_3 \in \mathbb{R}
\]


Resuelvo para $k = -1$:
\[
\left(  
\begin{array}{ccc|c}
1 & -1 & -1 & 0\\
-1 & 1 & 1 & 0\\
1 & -1 & -3 & 0
\end{array}
\right)
\xrightarrow[\substack{F_3 - F_1}]{F_2 + F_1}
\]

\[
\left(  
\begin{array}{ccc|c}
1 & 1 & -1 & 0\\
0 & 0 & 0 & 0\\
0 & 0 & -2 & 0\\
\end{array}
\right)
\]

Despejo las variables:
\begin{align*}
-2x_3 &= 0 \quad \Rightarrow \quad x_3 = 0 \\
x_1 + x_2 - (0) &= 0 \quad \Rightarrow  
\quad x_1 = -x_2
\end{align*}    
Por lo tanto, la solución general del sistema homogéneo para $k = -1$ es:\[
\begin{pmatrix}
-x_2 \\
x_2 \\
0 \\
\end{pmatrix}   =
x_2
\begin{pmatrix}
-1 \\
1 \\
0 \\
\end{pmatrix}, \quad x_2 \in \mathbb{R}
\]

\section*{Ejercicio 5}
\includegraphics[scale=0.6]{Consignas-imagenes/ejercicio5.png}

\subsection*{a)}
tengo x+y-z=0 y x-y=0. 
Esto forma la matriz:
\[
\left(
\begin{array}{ccc|c}    
1 & 1 & -1 & 0\\
1 & -1 & 0 & 0\\
\end{array}
\right)
\xrightarrow[\substack{F_2 - F_1}]{triangulo}
\]

\[
\left(  
\begin{array}{ccc|c}
1 & 1 & -1 & 0\\
0 & -2 & 1 & 0\\
\end{array}
\right)
\]

Ahora despejo las variables desde abajo hacia arriba:
\begin{align*}
-2y + z &= 0 \quad \Rightarrow \quad z = 2y \\
x + y - (2y) &= 0 \quad \Rightarrow \quad x - y = 0 \quad \Rightarrow \quad x = y
\end{align*}
Por lo tanto, la solución general del sistema es:\[
\begin{pmatrix}
y \\
y \\
2y \\
\end{pmatrix}   =
y
\begin{pmatrix}
1 \\
1 \\
2 \\
\end{pmatrix}, \quad y \in \mathbb{R}
\]

Verifico que este vector cumple ambas ecuaciones y, por lo tanto, sea un generador del subespacio:
\begin{align*}
x + y - z &= 1 + 1 - 2 = 0 \\
x - y &= 1 - 1 = 0
\end{align*}
Entonces, el vector $\begin{pmatrix} 1 \\ 1 \\ 2 \end{pmatrix}$ es un generador del subespacio.

\subsection*{b)}
Decir que una matriz $A$ cumple
\[
-A^t = A
\]
significa que la matriz es \textbf{antisimétrica}.

Como $A^t$ denota la traspuesta de $A$, la igualdad anterior es equivalente a
\[
A^t = -A.
\]

Esto implica las siguientes propiedades:

\begin{itemize}
    \item Para todo par de índices $i,j$ se cumple:
    \[
    a_{ij} = -a_{ji}.
    \]

    \item Los elementos de la diagonal principal son cero, ya que si $i=j$:
    \[
    a_{ii} = -a_{ii} \Rightarrow a_{ii} = 0.
    \]

    \item La matriz $A$ debe ser cuadrada.
\end{itemize}

Un ejemplo de matriz antisimétrica es:
\[
A =
\begin{pmatrix}
0 & 2 & -1 \\
-2 & 0 & 4 \\
1 & -4 & 0
\end{pmatrix}.
\]

Su traspuesta es:
\[
A^t =
\begin{pmatrix}
0 & -2 & 1 \\
2 & 0 & -4 \\
-1 & 4 & 0
\end{pmatrix},
\]
y se verifica que
\[
A^t = -A.
\]

Volviendo al ejercicio, tengo la siguiente matriz genérica de 3x3:
\[
A =
\begin{pmatrix}
a_{11} & a_{12} &  a_{13} \\
a_{21} & a_{22} & a_{23} \\
a_{31} & a_{32} & a_{33}
\end{pmatrix},
\]

La condición "$-A^t = A$" implica que:
\[
\begin{pmatrix}
-a_{11} & -a_{21} & -a_{31} \\
-a_{12} & -a_{22} & -a_{32} \\
-a_{13} & -a_{23} & -a_{33}
\end{pmatrix} =
\begin{pmatrix}
a_{11} & a_{12} &  a_{13} \\
a_{21} & a_{22} & a_{23} \\
a_{31} & a_{32} & a_{33}
\end{pmatrix}.
\]
Esto me da el siguiente sistema de ecuaciones:
\begin{align*}
-a_{11} &= a_{11}=0 \\
-a_{22} &= a_{22}=0 \\
-a_{33} &= a_{33}=0 \\
-a_{21} &= a_{12} \\
-a_{31} &= a_{13} \\
-a_{32} &= a_{23}
\end{align*}

Resolviendo este sistema, obtengo:
\[
A =
\begin{pmatrix}
0 & a_{12} &  a_{13} \\
-a_{12} & 0 & a_{23} \\
-a_{13} & -a_{23} & 0
\end{pmatrix},
\]
donde $a_{12}, a_{13}, a_{23} \in \mathbb{C}$ son parámetros libres.
Entonces, mis generadores del subespacio son:

\[
\begin{pmatrix}
0 & 1 & 0 \\
-1 & 0 & 0 \\
0 & 0 & 0
\end{pmatrix},
\quad
\begin{pmatrix}
0 & 0 & 1 \\
0 & 0 & 0 \\
-1 & 0 & 0
\end{pmatrix},
\quad
\begin{pmatrix}
0 & 0 & 0 \\
0 & 0 & 1 \\
0 & -1 & 0
\end{pmatrix}.
\]

\subsection*{c)}
Es un subespacio de las matrices 3×3 donde la única restricción es que la suma de la diagonal sea cero.


Una matriz genérica de 3x3 es:
\[
A =
\begin{pmatrix}
a_{11} & a_{12} &  a_{13} \\
a_{21} & a_{22} & a_{23} \\
a_{31} & a_{32} & a_{33}
\end{pmatrix},
\]
La condición "la suma de la diagonal es cero" implica que:
\begin{align*}
a_{11} + a_{22} + a_{33} &= 0 \quad \Rightarrow \quad a_{33} = -a_{11} - a_{22}
\end{align*}
Por lo tanto, la matriz queda de la siguiente forma:
\[
A =
\begin{pmatrix}
a_{11} & a_{12} &  a_{13} \\
a_{21} & a_{22} & a_{23} \\
a_{31} & a_{32} & -a_{11} - a_{22}
\end{pmatrix},
\]

donde $a_{11}, a_{12}, a_{13}, a_{21}, a_{22}, a_{23}, a_{31}, a_{32} \in \mathbb{R}$ son parámetros libres.
Entonces, mis generadores del subespacio son:
\[
\begin{pmatrix}
1 & 0 & 0 \\
0 & 0 & 0 \\
0 & 0 & -1
\end{pmatrix},
\quad
E_{12}
\quad
E_{13},
\quad
E_{21},
\quad
\begin{pmatrix}
0 & 0 & 0 \\
0 & 1 & 0 \\
0 & 0 & -1
\end{pmatrix},
\quad
E_{23},
\quad
E_{31},
\quad
E_{32}
\]

\subsection*{d)}

Sea el subespacio de $\mathbb{C}^4$ definido por el siguiente sistema de ecuaciones lineales:
\[A=
\left(
\begin{array}{cccc|c}
    1&1&0&-i&0\\
    i&1+i&-1&0&0\\
\end{array}
\right)
\xrightarrow[\substack{F_2 - iF_1}]{}
\]

\[
\left(
\begin{array}{cccc|c}
    1&1&0&-i&0\\
    0&1&-1&-1&0\\
\end{array}
\right)
\]

Entonces, despejo las variables desde abajo hacia arriba:
\begin{align*}
x_2 - x_3 - x_4 &= 0 \quad \Rightarrow \quad x_2 = x_3 + x_4 \\
x_1 + (x_3 + x_4) - ix_4 &= 0 \quad \Rightarrow \quad x_1 + x_3 + x_4 - ix_4 = 0 \quad \Rightarrow \quad x_1 = -x_3 - x_4 + ix_4
\end{align*}
Por lo tanto, la solución general del sistema es:\[
\begin{pmatrix}
    -x_3 - x_4 + ix_4 \\
    x_3 + x_4 \\
    x_3 \\
    x_4 \\
\end{pmatrix}   =
x_3
\begin{pmatrix}
    -1 \\
    1 \\
    1 \\
    0 \\
\end{pmatrix} +
x_4
\begin{pmatrix}
    -1 + i \\
    1 \\
    0 \\
    1 \\
\end{pmatrix}, \quad x_3, x_4 \in \mathbb{C}
\]

y los generadores del subespacio son:
\[
\begin{pmatrix}
    -1 \\
    1 \\
    1 \\
    0 \\
\end{pmatrix},
\quad
\begin{pmatrix}
    -1 + i \\
    1 \\
    0 \\
    1 \\
\end{pmatrix}
\]

Verificamos:
\begin{align*}
x_1 + x_2 - ix_4 &= -1 + 1 + 1 - i(0) = 0 \\
ix_1 + (1 + i)x_2 - x_3 &= i(-1) + (1 + i)(1) - (1) = -i + 1 + i - 1 = 0
\end{align*}

y 
\begin{align*}
x_1 + x_2 - ix_4 &= (-1 + i) + 1 + 0 - i(1) = -1 + i + 1 - i = 0 \\
ix_1 + (1 + i)x_2 - x_3 &= i(-1 + i) + (1 + i)(1) - 0 = -i + i^2 + 1 + i = -1 + 1 = 0
\end{align*}


\includegraphics[scale=0.6]{Consignas-imagenes/Captura de pantalla 2026-01-26 193118.png}

\subsection*{a)}
Para determinar si el vector (2,1,3,5) $\epsilon$ $S$=$\langle(1,-1,2,1),(3,1,0,-1),(1,1,-1,-1)\rangle$, necesitamos verificar si existen escalares $a$, $b$ y $c$ tales que:
\[a(1,-1,2,1) + b(3,1,0,-1) + c(1,1,-1,-1) = (2,1,3,5)\]
Esto nos lleva al siguiente sistema de ecuaciones:
\begin{align*}
a + 3b + c &= 2 \quad (1)\\
-a + b + c &= 1 \quad (2)\\
2a + 0b - c &= 3 \quad (3)\\
a - b - c &= 5 \quad (4)
\end{align*}
Resolvemos este sistema de ecuaciones. Sumando (2) y (4):
\begin{align*}
(-a + b + c) + (a - b - c) &= 1 + 5 \\
0 &= 6
\end{align*}
Esto es una contradicción, lo que significa que no existen tales escalares $a$, $b$ y $c$. Por lo tanto, el vector (2,1,3,5) no pertenece al subespacio $S$.

\subsection*{b)}
Para determinar si $\{x \in \mathbb{R}^4 \mid x_1 - x_2 - x_3 = 0\} \subseteq S$ , necesitamos verificar si cualquier vector que satisface la ecuación $x_1 - x_2 - x_3 = 0$ puede ser expresado como una combinación lineal de los vectores generadores de $S$.
Sea $x = (x_1, x_2, x_3, x_4)$ tal que $x_1 - x_2 - x_3 = 0$. Entonces, podemos expresar $x_1$ en términos de $x_2$ y $x_3$:
\[x_1 = x_2 + x_3\]
Queremos encontrar si existen escalares $a$, $b$ y $c$ tales que:
\[a(1,-1,2,1) + b(3,1,0,-1) + c(1,1,-1,-1) = (x_1, x_2, x_3, x_4)\]
Esto nos lleva al siguiente sistema de ecuaciones:
\begin{align*}
a + 3b + c &= x_1 \quad (1)\\
-a + b + c &= x_2 \quad (2)\\
2a + 0b - c &= x_3 \quad (3)\\
a - b - c &= x_4 \quad (4)
\end{align*}
Sustituyendo $x_1$ en (1):
\begin{align*}
a + 3b + c &= x_2 + x_3 \quad (1')
\end{align*}
Sumando (2) y (4):
\begin{align*}
(-a + b + c) + (a - b - c) &= x_2 + x_4 \\
0 &= x_2 + x_4
\end{align*}
Esto implica que $x_4 = -x_2$. Por lo tanto, no todos los vectores que satisfacen $x_1 - x_2 - x_3 = 0$ pueden ser expresados como combinaciones lineales de los vectores generadores de $S$. Por lo tanto, $\{x \in \mathbb{R}^4 \mid x_1 - x_2 - x_3 = 0\} \not\subseteq S$.
Esto se debe a que  $x_4$ se supone que no tiene ninguna restricción y es libre, lo que se contradice con lo que encontramos.

\subsection*{c)}
Para determinar si $S \subseteq \{x \in \mathbb{R}^4 \mid x_1 - x_2 - x_3 = 0\} $, necesitamos verificar si cualquier combinación lineal de los vectores generadores de $S$ satisface la ecuación $x_1 - x_2 - x_3 = 0$.
Sea $x = a(1,-1,2,1) + b(3,1,0,-1) + c(1,1,-1,-1)$ para algunos escalares $a$, $b$ y $c$. Entonces, tenemos:
\begin{align*}  
x_1 &= a + 3b + c \\
x_2 &= -a + b + c \\
x_3 &= 2a + 0b - c \\
\end{align*}
Calculamos $x_1 - x_2 - x_3$:
\begin{align*}
x_1 - x_2 - x_3 &= (a + 3b + c) - (-a + b + c) - (2a + 0b - c) \\
&= a + 3b + c + a - b - c - 2a - 0b + c \\
&= (a + a - 2a) + (3b - b - 0b) + (c - c + c) \\
&= 0 + 2b + c
\end{align*}

Para que $x_1 - x_2 - x_3 = 0$, necesitamos que $2b + c = 0$. Esto significa que no todos los vectores en $S$ satisfacen la ecuación $x_1 - x_2 - x_3 = 0$ a menos que se imponga la restricción adicional de que $c = -2b$.
Por lo tanto, $S \not\subseteq \{x \in \mathbb{R}^4 \mid x_1 - x_2 - x_3 = 0\}$.
Si hubiera dado 0 en lugar de $2b + c$ en el cálculo, entonces hubiera sido cierto.

\includegraphics[scale=0.5]{Consignas-imagenes/Captura de pantalla 2026-01-26 203005.png}
\includegraphics[scale=0.5]{Consignas-imagenes/Captura de pantalla 2026-01-26 203404.png}


















\end{document}